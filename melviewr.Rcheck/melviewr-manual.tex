\nonstopmode{}
\documentclass[a4paper]{book}
\usepackage[times,inconsolata,hyper]{Rd}
\usepackage{makeidx}
\usepackage[utf8]{inputenc} % @SET ENCODING@
% \usepackage{graphicx} % @USE GRAPHICX@
\makeindex{}
\begin{document}
\chapter*{}
\begin{center}
{\textbf{\huge Package `melviewr'}}
\par\bigskip{\large \today}
\end{center}
\begin{description}
\raggedright{}
\inputencoding{utf8}
\item[Type]\AsIs{Package}
\item[Title]\AsIs{View and Classify MELODIC Output for ICA+FIX}
\item[Version]\AsIs{0.0.0.9000}
\item[Author]\AsIs{Andrew Poppe}
\item[Maintainer]\AsIs{The package maintainer }\email{yourself@somewhere.net}\AsIs{}
\item[Description]\AsIs{More about what it does (maybe more than one line)
Use four spaces when indenting paragraphs within the Description.}
\item[License]\AsIs{GPL-3}
\item[Encoding]\AsIs{UTF-8}
\item[LazyData]\AsIs{true}
\item[Imports]\AsIs{gtools, RColorBrewer, RNifti, grDevices, RGtk2, cairoDevice,
methods}
\item[Depends]\AsIs{gWidgetsRGtk2, gWidgets}
\item[RoxygenNote]\AsIs{5.0.1}
\item[NeedsCompilation]\AsIs{no}
\end{description}
\Rdcontents{\R{} topics documented:}
\inputencoding{utf8}
\HeaderA{melviewr-package}{melviewr: A viewer for MELODIC output and ICA+FIX classification.}{melviewr.Rdash.package}
%
\begin{Description}\relax
The melviewr package allows the user to easily view and classify
MELODIC output for the purposes of later running ICA+FIX. The user
categorizes a component as signal or noise based on its spatial
characteristics as well as its temporal profile. melviewr can then save
a text file of these classifications in the format required by ICA+FIX.
\end{Description}
%
\begin{Section}{melviewr functions}

melviewr
\end{Section}
\inputencoding{utf8}
\HeaderA{melviewr}{melviewr}{melviewr}
%
\begin{Description}\relax
View and Classify Components from a Melodic Analysis
\end{Description}
%
\begin{Usage}
\begin{verbatim}
melviewr(melodic_dir, standard_file = NULL, motion_file = NULL)
\end{verbatim}
\end{Usage}
%
\begin{Arguments}
\begin{ldescription}
\item[\code{melodic\_dir}] string Path to MELODIC output directory

\item[\code{standard\_file}] string Optional path to a 3-dimensional Nifti standard file
of the same voxel dimensions as the melodic output

\item[\code{motion\_file}] string Optional path to a summary motion text file. This file
should have one column and as many rows as there are volumes in the functional
data
\end{ldescription}
\end{Arguments}
\printindex{}
\end{document}
